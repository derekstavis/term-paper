% ==================================================================== %
Resumo % ====================================================================

\noindent STAVIS, Derek Willian. \textbf{Uma proposta de interface do cliente
de torrent Transmission baseado no GTK HIG 3.14} Florianópolis, 2013.
\pageref{nropaginas}f. Trabalho de Conclusão de Curso Superior de Tecnologia
em Análise e Desenvolvimento de Sistemas. Faculdade de Tecnologia do SENAI,
Florianópolis, 2013.

\vspace{1cm}
\setlength{\absparsep}{18pt} % ajusta o espaçamento dos parágrafos do resumo
\begin{resumo}

 \textbf{Palavras-chave}: Interfaces Gráficas. GTK+. GNOME. Design. Redesign.
 
  Os avanços trazidos nos últimos quatro anos no ambiente gráfico GNOME foram de
  suma importância para atingir uma experiência diferenciada e consistente de
  uso. Conceitos já difundidos no design de interfaces de usuário foram
  revisitados, paradigmas foram quebrados, mudando a forma com que os usuários
  vêem e interagem com o sistema operacional. Entretanto, nem todos os softwares
  disponíveis para a plataforma acompanharam a velocidade de desenvolvimento da
  plataforma GNOME, e muitos ainda necessitam de atenção.
  
  Com o lançamento do HIG (Guia de Interface com o Usuário) no GNOME versão 3.12
  toda filosofia e padrões de design da plataforma foram sintetizados em
  linguagem simples e objetiva, tornando-se o material referência para o
  desenvolvimento e manutenção de seu ecosistema de softwares. Esta pesquisa
  aborda o processo de adaptação do cliente de torrent Transmission de acordo
  com o guia de interfaces do GNOME versão 3.14.

 
\end{resumo}
