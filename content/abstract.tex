\noindent STAVIS, Derek Willian. \textbf{\imprimirtitulo} Florianópolis, 2015.
\pageref{nropaginas}f. Trabalho de Conclusão de Curso Superior de Tecnologia em
Análise e Desenvolvimento de Sistemas. Faculdade de Tecnologia do SENAI,
Florianópolis, 2015.

\vspace{1cm}
\begin{resumo}[\textbf{ABSTRACT}]
 \begin{otherlanguage*}{english}

  The advances in the last four years in GNOME Desktop Environment was utmost to
  achieve a consistent and distinguished user experience. Widespread concepts
  already used in user interface design were revisited and paradigms were
  broken, thus changing how the user see and interact with a computer. Although,
  not every software available for the GNOME desktop followed with the
  transition period, and most softwares needs attention. With the release of
  GNOME Human Interface Guidelines together with version 3.12 of GNOME as a
  official documentation for design principles, all platform's philosophies and
  design patterns were translated in a simple and objective language, aiming to
  be the reference for both development andmaintenance of GNOME's ecosystem.
  This research addresses the thinking process of redesigning the user interface
  of Transmission, a free and open source client for file sharing, targeting
  GNOME HIG version 3.14.

   \vspace{\onelineskip}

   \noindent
   \textbf{Key-words}: Graphical User Interfaces. GTK+. GNOME. Design. Redesign.
 \end{otherlanguage*}
\end{resumo}
