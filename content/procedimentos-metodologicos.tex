\chapter{PROCEDIMENTOS METODOLÓGICOS}

Em um primeiro instante, com o objetivo de avaliar o processo de evolução da
plataforma GNOME e seu design, foram identificadas aplicações centrais do
GNOME 3.16 que já implementam os padrões de design definidos no HIG.

Pelo importante papel desempenhado na experiência GNOME, foram selecionadas três
aplicações centrais:

\begin{itemize}
    \item Nautilus -- Navegador de Arquivos
    \item Evince -- Visualizador de Documentos Digitais
    \item Gedit -- Editor de Arquivos de Texto
\end{itemize}

\section{Análise cronológica dos padrões de design nas aplicações}
\label{sec:chronologic-analysis}

Uma análise cronológica foi elaborada através da comparação de duas versões
distantes do GNOME e das aplicações centrais escolhidas. Os parâmetros de
análise foram escolhidos dentro dos padrões de design especificados pelo HIG
\cite{hig314patterns}:

\begin{enumerate}
  \item Menu da Aplicação
  \item Janela Primária
  \item Barra de Título
  \item Comutador de Visão
  \item Busca
\end{enumerate}

O processo de análise cronológica foi facilitado pela disponibilidade de imagens
oficiais de um sistema operacional com todo software necessário para uma
experiência GNOME básica.

Detalhadas na tabela \ref{gnome-versions} estão as informações das versões
escolhidas, dentre as disponíveis para análise \cite {gnome2015promo-usb}.

\citeonline{popper1959logic} afirma que no ramo da pesquisa experimental é de
suma importância a reprodutibilidade de um experimento, argumentando inclusive
que ocorrências não reproduzíveis não tem significância para a ciência.

A soma de verificação de cada arquivo foi utilizada para checar a integridade do
sistema operacional, e todas aplicações foram analisadas na versão oficial --
sem alterações na configuração do sistema, troca de temas, fontes, etc.

\begin{center}
    \begin{tabularx}{\textwidth}{ | l | l | l | X | }
    \hline
    Versão do GNOME & Data de Modificação & Nome do arquivo & Soma de Verificação \\
    \hline
    3.6.0 & 08/10/2012 & GNOME-3.6.0.iso       & MD5    753c99ce2342f658
                                                        65c1f74bc3722e44 \\
    \hline
    3.16  & 25/03/2015 & gnome-3.16.x86-64.iso & SHA256 4a6185a0aca89f15
                                                        8f769d76d5a0086f
                                                        0f1e9d709a5d80cd
                                                        cf2b0d52d67ab2b2 \\
    \hline
    \end{tabularx}
\end{center}

A execução das imagens se deu em ambiente virtual, provisionado utilizando o
software Oracle VM VirtualBox 4.3.26 com a seguinte configuração de máquina:

\begin{center}
    \begin{tabular}{ | l | l | }
    \hline
    \multicolumn{2}{| l |}{Máquina} \\
    \hline
    Memória Base  & 4096MB \\
    Processadores & 2 \\
    Aceleração    & VT-x/AMD-V, Nested Paging, PAE/NX \\
    \hline
    \multicolumn{2}{| l |}{Vídeo} \\
    \hline
    Memória       & 128MB \\
    Aceleração    & 3D \\
    \hline
    \multicolumn{2}{| l |}{Sistema Operacional} \\
    \hline
    Tipo         & Linux \\
    Versão       & Fedora \\
    \hline
    \end{tabular}
\end{center}

O registro gráfico das aplicações analisadas se deu através de captura de telas,
quais podem ser encontradas na íntegra no apêndice \ref{app:chrono-screenshots}.

% TODO: Falar mais sobre a metodologia de análise das aplicações centrais.
% Foram:
% - Comparadas as capturas de tela
% - Relacionados componentes gráficos equivalentes aos escolhidos
% - Analise de uso de espaço

Pelas diferenças existentes na análise de caráter cronológico, os padrões de
design escolhidos foram relacionados nas duas versões analisadas, transportando,
quando necessário, a forma e função de um padrão presente mais antigo com o
padrão analisado.

\section{Identificação de pontos de redesign na janela principal do Transmission}

Utilizando-se do mesmo ambiente de testes e procedimentos de aquisição de dados
estabelecidos na seção \ref{sec:chronologic-analysis} foram feitos registros
gráficos do Transmission na versão mais recente do GNOME. Migrações de padrões
de design identificados foram transportados para o contexto do Transmission.

% TODO: Tweak this

\section{Proposição de melhorias na interface do Transmission}

Com os resultados prévios desta pesquisa e o HIG em mãos foram elencados pontos
de retrabalho na interface do Transmission. A documentação oficial e fatores
como utilização prévia por ambos aplicativos centrais e outras versões do
Transmission foram levados em consideração em sua formulação.

A prototipação de interfaces é uma técnica de evolução de design que tem como
base a constante interação e avaliação dos resultados \cite{de2009client}.
\citeonline[p. 2]{baumer1996user} separa em quatro classes os protótipos de
interfaces gráficas:

\begin{enumerate}
  \item De apresentação
  \item Funcional
  \item Experimental
  \item Piloto
\end{enumerate}

A utilização de protótipos funcionais permite implementar estrategicamente ambas
interface gráfica e funcionalidade de um programa para averiguar o funcionamento
de um conceito.

Sendo o Transmission um software de código aberto, licenciado nos termos da GPL,
seu código fonte é divulgado publicamente e pode ser alterado para fins de
estudo.

Com os pontos de retrabalho em mãos foi produzido um protótipo funcional para
averiguar a validade das propostas e sua integração na plataforma GNOME.
