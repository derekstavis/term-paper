\chapter{PROCEDIMENTOS METODOLÓGICOS}

Em um primeiro instante, com o objetivo de avaliar o processo de evolução da
plataforma GNOME e seu design, foram identificadas aplicações centrais do
GNOME 3.16 que já implementam os padrões de design definidos no HIG. Pelo
importante papel desempenhado na experiência GNOME, foram selecionadas três
aplicações centrais:

\begin{itemize}
    \item Nautilus -- Navegador de Arquivos
    \item Evince -- Visualizador de Documentos Digitais
    \item Gedit -- Editor de Arquivos de Txewhto
\end{itemize}

\section{Análise cronológica dos padrões de design}

Com inuito de averiguar os padrões de design especificados pelo HIG
\cite{hig314patterns} foram definidos alguns parâmetros a serem observados na
evolução do design dos aplicativos analisados.

\begin{enumerate}
  \item Menu da Aplicação
  \item Janela Primária
  \item Barra de Título
  \item Comutador de Visão
  \item Busca
\end{enumerate}

\section{Configuração do Ambiente de Testes}

O projeto GNOME disponibiliza imagens oficiais de um sistema operacional Linux
com todos os pacotes necessários para uma experiência GNOME pré instados e
configurados. Sua utilização trouxe facilidade e consistência nos testes
efetuados.

Dentre as versões disponíveis foram selecionadas :

\begin{center}
    \begin{tabularx}{\textwidth}{ | l | l | l | X | }
    \hline
    Versão do GNOME & Data de Modificação & Nome do arquivo & Soma de Verificação \\
    \hline
    3.6.0 & 08/10/2012 & GNOME-3.6.0.iso       & MD5    753c99ce2342f658
                                                        65c1f74bc3722e44 \\
    \hline
    3.16  & 25/03/2015 & gnome-3.16.x86-64.iso & SHA256 4a6185a0aca89f15
                                                        8f769d76d5a0086f
                                                        0f1e9d709a5d80cd
                                                        cf2b0d52d67ab2b2 \\
    \hline
    \end{tabularx}
\end{center}

Todas aplicações foram analisadas na versão publicada, sem alterações na
configuração do sistema, troca de temas, fontes, etc. O ambiente virtual foi
provisionado utilizando o software Oracle VM VirtualBox 4.3.26 na seguinte
configuração de máquina:

\begin{center}
    \begin{tabular}{ | l | l | }
    \hline
    \multicolumn{2}{| l |}{Máquina} \\
    \hline
    Memória Base  & 4096MB \\
    Processadores & 2 \\
    Aceleração    & VT-x/AMD-V, Nested Paging, PAE/NX \\
    \hline
    \multicolumn{2}{| l |}{Vídeo} \\
    \hline
    Memória       & 128MB \\
    Aceleração    & 3D \\
    \hline
    \multicolumn{2}{| l |}{Sistema Operacional} \\
    \hline
    Tipo         & Linux \\
    Versão       & Fedora \\
    \hline
    \end{tabular}
\end{center}

