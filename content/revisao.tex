O desenvolvimento de mecanismos que automatizam tarefas existe desde a idade das
pedras, onde hominídeos produziam ferramentas para auxílio próprio. A
transversalidade do conhecimento e da experimentação nos levou a descoberta de
novas metodologias e o aperfeiçoamento de técnicas, consequentemente modificando
a forma com que o ser humano interage com o mundo \cite[p.
1]{sartori2010neurociencia}.

\section{Interfaces gráficas}

O conceito de interface é extremamente amplo, e foi largamente difundido com o
início dos estudos de interação humano-máquina. O avanço da industrialização
permitiu com que máquinas extremamente complexas substituíssem seres humanos nas
tarefas mais difíceis,  deixando estes, seus operadores, apenas com a
responsabilidade de pilotá-las de forma simples e segura.

O advento de tecnologias multimídias interativas como computadores, celulares e
tablets elevou o patamar da criação de interfaces e gestão de tarefas ao estado
da arte, agregando conhecimentos transversais de artistas visuais a músicos.

\citeonline[p. 1]{ferreira2005semiotics} afirmam que a criação de “interfaces
com o usuário ainda está mais para arte do que para ciência.", e justificam,
explicando que "A maior parte do design ou redesign é baseado em estudos
empíricos ou protótipos, e ainda há muito pouca compreensão teórica ou de
engenharia de como conduzir o processo de design e produzir bons designs pela
primeira vez”.

O primeiro conceito científico de gerenciamento de tarefas através da
sobreposição de janelas data de 1969, na tese de Ph.D. de Alan Kay. A
implementação de seu conceito, vista pela primeira vez em funcionamento no
sistema do Xerox PARC, é largamente utilizada até hoje por grandes sistemas
tanto comerciais quanto de código aberto, como Windows, OS X, GNOME, KDE
\cite[p. 7]{myers2000past}.

Interfaces gráficas modernas foram alcançadas através da associação entre
hardware e software, utilizando de diversos dispositivos de entrada e saída,
além de  inúmeras camadas de software, com o intuito de promover usabilidade e
fácil adaptação.

Dispositivos de entrada transformam coordenadas do mundo real para o mundo
virtual, registrando condições externas que podem ser modificadas através da
interação com um ou mais atores. Dispositivos de entrada comuns são mouse, tela
de toque, teclado, etc.

Dispositivos de saída projetam informações geradas por um sistema computacional
e seu caráter é geralmente baseado nos sentidos: Visão, audição, tato. O
dispositivo de saída mais comum é o monitor, que tem por finalidade projetar
imagens compatíveis com a capacidade humana de visão.

\section{Gerenciadores de Janelas e Toolkits}

Uma das responsabilidades de um gerenciadores de janelas é disponibilizar ou
consumir áreas de desenho e exibí-las em um dispositivo de saída de vídeo. Estas
áreas de desenho são chamadas de janelas.

Um gerenciador de janelas moderno também tem por finalidade coordenar a exibição
deste conjunto de janelas em um conjunto de monitores, escutar por eventos de
entrada (mouse, teclado) e informar os responsáveis pelas janelas sobre
alterações no layout de tela (dimensões da tela, dimensões da janela, espaço de
cor).

Gerenciadores de janelas, porém, não tem por responsabilidade preencher as áreas
de desenho com gráficos, e como suas APIs operam geralmente a nível de pixel, a
tarefa de escrever um programa gráfico acaba sendo demorada e entediante.
\apudonline{rosenthal1988simple}{myers2000past} Além disso, se cada
desenvolvedor criasse seus próprios componentes, seria praticamente impossível
disponibilizar uma experiência consistente ao usuário.

Para solucionar este problema ferramentas conhecidas como toolkits foram criadas
sobre as abstrações disponibilizadas pelos WMs. Sua finalidade é esconder
características de baixo nível, disponibilizando uma fachada homogênea e
portável para o desenvolvedor de software, além de comportamento e experiência
visual consistente para o usuário final.

As responsabilidades de um toolkit inclui desenhar elementos de interface
gráfica como texto, botões, imagens, barras de progresso, etc, de acordo com um
ou mais estilos visuais. Também é sua responsabilidade processar eventos de um
ou mais dispositivos de entrada (mouse, teclado, painel de toque) verificar a
colisão de um evento com elementos de interface gráfica (clique em um botão, por
exemplo), traduzir e informar os eventos para a aplicação proprietária da
janela.

Além disso, toolkits multiplataforma são utilizados para escrever interfaces
gráficas portáveis, permitindo com que o mesmo código seja recompilado para um
sistema operacional diferente do em que foi escrito e funcione da mesma forma.

\section{A plataforma GNOME e o toolkit GTK}

Existem diversas opções de gerenciadores de janela de código aberto, comumente
incluídos em várias distribuições de Linux. O GNOME, um ambiente gráfico
bastante difundido pelos usuários de Linux, foi fundado e está em constante
desenvolvimento por uma comunidade de engenheiros de software ao redor do mundo.

Muito além de um mero gerenciador de janelas, a plataforma GNOME se desenvolveu
a ponto de ser constituída por uma série de aplicações base, incluindo um
gerenciador de janelas, um lançador de aplicações e diversos aplicativos
integrados, como calculadora, editor de texto, gerenciadores de arquivos, redes,
contatos, etc.

A plataforma GNOME e seus aplicativos integrados são altamente baseados no
toolkit GTK. De acordo com o site oficial, o ”GTK +, ou GIMP Toolkit, é um
conjunto de ferramentas multi-plataforma para criar interfaces gráficas de
usuário. Oferecendo um conjunto completo de widgets, o GTK + é adequado para
projetos desde pequenas ferramentas pontuais até suítes completas de
aplicativos.”  [1].

\section{O código aberto e desenvolvimento contínuo}

Uma das característica das plataformas de código aberto é a distribuição de
esforços em prol do constante desenvolvimento e melhoria. A pluralidade de
opiniões e idéias eleva o patamar das discussões e permite com que vários pontos
de vista sejam levados em consideração na evolução da plataforma.

Apesar dos prós existentes na distribuição de esforços também existem os contras
— Projetos que são desenvolvidos paralelamente nem sempre avançam na mesma
velocidade. E o contra fica mais sério quando um projeto depende do outro, como
é o caso de toolkits e programas que consomem suas APIs (Application Programming
Interfaces - Interfaces Programador-Programa).

Além dos aplicativos integrados, geralmente garantidos de acompanhar a evolução
da plataforma, sua ergonomia e visual, existe uma vasta gama de aplicações tanto
de código aberto quanto proprietárias disponíveis para atender as mais variadas
necessidades. Em sua grande maioria os aplicativos também são mantidos pela
comunidade, e se não atualizados podem ficar defasados em usabilidade, ergonomia
e consistência com a plataforma.

\section{Impactos na consistência e usabilidade}

Shneiderman, 1992 descreve a usabilidade como uma combinação das seguintes
características:

\begin{itemize}   \item Facilidade de aprendizado   \item Alta velocidade de
operação   \item Baixa taxa de erros   \item Satisfação do usuário   \item
Retenção de usuários pelo tempo \end{itemize}

Em prol da usabilidade a maioria dos toolkits aplica um modelo próprio de
apresentação, organização, interação e estilização de widgets, que
costumeiramente varia de acordo com o tipo de ambiente onde é executado
(desktop, tablet e celular).

O conjunto de padrões de design recomendado ao utilizar o GTK para garantir a
máxima usabilidade da plataforma foi especificado oficialmente através de um
documento lançado no ano de 2014, acompanhando a versão 3.14 do toolkit, sob o
título de Human Interface Guidelines.

O HIG, como também é chamado, é uma literatura ilustrada de diretrizes
recomendadas no desenvolvimento de interfaces gráficas que utilizem o toolkit,
com o intuito de reforçar a consistência visual e integração com diferentes
gerenciadores de janela.

\section{Estudo de caso: Transmission}

Um aplicativo famoso disponível para a plataforma GNOME é o cliente de
BitTorrent Transmission. Aclamado pela sua simplicidade, o aplicativo é
utilizado para compartilhar arquivos através da internet.

A interface gráfica do Transmission foi desenvolvida na era GNOME 2, e por mais
que componentes básicos como botões e listas não tenham mudado agressivamente no
GNOME 3, muitos padrões de design foram criados ou mudaram, trazendo mais
conforto e consistência para os usuários e desenvolvedores do toolkit.

O objetivo principal desta pesquisa é explorar o HIG do GTK 3.14, além de
avançar o desenvolvimento de um software livre mantido pela comunidade de código
aberto, propondo e contribuindo melhorias visuais e de código-fonte na interface
gráfica do Transmission.

Esta pesquisa foi baseada na análise de diversos casos de uso de softwares
contidos na plataforma GNOME Desktop, versão 3.14, e tem a pretensão de
descrever o processo de pensamento e a motivação por trás das adaptações de
interface gráficas propostas.
