\documentclass[
	% -- opções da classe memoir --
	12pt,				% tamanho da fonte
	openright,			% capítulos começam em pág ímpar (insere página vazia caso preciso)
	oneside,			% para impressão em verso e anverso. Oposto a oneside
	a4paper,			% tamanho do papel. 
	% -- opções da classe abntex2 --
	chapter=TITLE,		% títulos de capítulos convertidos em letras maiúsculas
	section=TITLE,		% títulos de seções convertidos em letras maiúsculas
	%subsection=TITLE,	% títulos de subseções convertidos em letras maiúsculas
	%subsubsection=TITLE,% títulos de subsubseções convertidos em letras maiúsculas
	% -- opções do pacote babel --
	brazil				% o último idioma é o principal do documento
	]{abntex2}

% ---
% PACOTES
% ---

% ---
% Pacotes fundamentais 
% ---
\usepackage{times}			% Usa a fonte Times
\usepackage[T1]{fontenc}		% Selecao de codigos de fonte.
\usepackage[utf8]{inputenc}		% Codificacao do documento (conversão automática dos acentos)
\usepackage{lastpage}			% Usado pela Ficha catalográfica
\usepackage{indentfirst}		% Indenta o primeiro parágrafo de cada seção.
\usepackage{color}			% Controle das cores
\usepackage{graphicx}			% Inclusão de gráficos
\usepackage{microtype} 			% para melhorias de justificação
\usepackage{listings}			% para inserir código fonte

% ---
% Pacotes de citações
% ---
\usepackage[brazilian,hyperpageref]{backref}	 % Paginas com as citações na bibl
\usepackage[alf]{abntex2cite}	% Citações padrão ABNT
\usepackage{titlesec}

% --- 
% CONFIGURAÇÕES DE PACOTES
% --- 

% altera o espacamento depois do número de cada secao, subsecao, etc
\titleformat{\section}
  {\normalfont\normalsize}{}{0pt}{\thesection\space}
\titleformat{\subsection}
  {\normalfont\normalsize\bfseries}{}{0pt}{\thesubsection\space}
\titleformat{\subsubsection}
  {\normalfont\normalsize}{}{0pt}{\thesubsubsection\space}
\titleformat{\paragraph}
  {\normalfont\normalsize\itshape}{}{0pt}{\theparagraph\space}

% ---
% Configurações do pacote backref
% Usado sem a opção hyperpageref de backref
\renewcommand{\backrefpagesname}{Citado na(s) página(s):~}
% Texto padrão antes do número das páginas
\renewcommand{\backref}{}
% Define os textos da citação
\renewcommand*{\backrefalt}[4]{
	\ifcase #1 %
		Nenhuma citação no texto.%
	\or
		Citado na página #2.%
	\else
		Citado #1 vezes nas páginas #2.%
	\fi}%
% ---

% ---
% **************************************************
% Informações que devem ser alteradas
% **************************************************
\titulo{\uppercase{Uma proposta de interface do cliente de torrent Transmission baseado no GTK HIG 3.14}}
\autor{\uppercase{Derek Willian Stavis}}
\orientador{Aline Cristina Antoneli de Oliveira}
\orientadortcc{Prof. \imprimirorientador, Dr. (SENAI/SC)}
\coordenador{Prof. Luciana Schmitz, Esp. (SENAI/SC)}
\coordenadortcc{Profa. Jaqueline Voltolini de Almeida, Me. (SENAI/SC)}
\examinador{Prof. Fulado de tal, Me. (SENAI/SC)}
\preambulo{Trabalho de Conclusão de Curso apresentado à Banca Examinadora do Curso Superior de Tecnologia em Análise e Desenvolvimento de Sistemas da Faculdade de Tecnologia do SENAI Florianópolis como requisito parcial para obtenção do Grau de Tecnólogo em Análise e Desenvolvimento de Sistemas.}
\proforientador{Professor Orientador: \imprimirorientador.}
\datadefesa{\uppercase{28 de Novembro de 2014}}
\local{Florianópolis/SC}
\data{2014}
% **************************************************

\instituicao{%
  SERVIÇO NACIONAL DE APRENDIZAGEM INDUSTRIAL 
  \par
  FACULDADE DE TECNOLOGIA SENAI/SC FLORIANÓPOLIS
  \par
  CURSO SUPERIOR DE TECNOLOGIA EM ANÁLISE E DESENVOLVIMENTO DE SISTEMAS}
\tipotrabalho{Trabalho de Conclusão de Curso}

% alterando o aspecto da cor azul
\definecolor{blue}{RGB}{41,5,195}

% informações do PDF
\makeatletter
\hypersetup{
     	%pagebackref=true,
		pdftitle={\@title}, 
		pdfauthor={\@author},
    	pdfsubject={\imprimirpreambulo},
	    pdfcreator={LaTeX with abnTeX2},
		pdfkeywords={abnt}{latex}{abntex}{abntex2}{trabalho acadêmico}, 
		colorlinks=true,      	% false: boxed links; true: colored links
		linkcolor=black,	% color of internal links
		citecolor=black,        % color of links to bibliography
		filecolor=magenta,      % color of file links
		urlcolor=black,
		bookmarksdepth=4
}
\makeatother
% --- 

% --- 
% Espaçamentos entre linhas e parágrafos 
% --- 

% O tamanho do parágrafo é dado por:
\setlength{\parindent}{1.3cm}

% Controle do espaçamento entre um parágrafo e outro:
\setlength{\parskip}{0.2cm}  % tente também \onelineskip

%\titlespacing\section{0pt}{12pt plus 4pt minus 2pt}{-6pt plus 2pt minus 2pt}

% ---
% compila o indice
% ---
\makeindex
% ---

% ----
% Início do documento
% ----
\begin{document}

% Retira espaço extra obsoleto entre as frases.
\frenchspacing 

% ----------------------------------------------------------
% ELEMENTOS PRÉ-TEXTUAIS
% ----------------------------------------------------------

\imprimircapa
\imprimirfolhaderosto*

\begin{folhadeaprovacao}

  \begin{center}
    {\ABNTEXchapterfont\bfseries\normalsize\imprimirautor}

    \vspace*{\fill}\vspace*{\fill}
    \begin{center}
      \ABNTEXchapterfont\bfseries\normalsize\imprimirtitulo
    \end{center}
    \vspace*{\fill}
        \imprimirpreambulo
    \vspace*{\fill}
        
   APROVADA PELA {\bfseries{COMISSÃO EXAMINADORA}}
   \par
   EM FLORIANÓPOLIS, \bfseries{\imprimirdatadefesa}
   \end{center}

   \assinatura{\imprimircoordenador \\ Coordenador do Curso} 
   \assinatura{\imprimircoordenadortcc \\ Coordenador de TCC} 
   \assinatura{\imprimirorientadortcc \\ Orientador} 
   \assinatura{\imprimirexaminador \\ Examinador}
   \begin{center}
    \vspace*{1cm}
  \end{center}
\end{folhadeaprovacao}

% Arquivos que devem ser alterados
% ====================================================================
% Dedicatória 
% ====================================================================
\begin{dedicatoria}

\end{dedicatoria}

% ====================================================================
% Agradecimentos 
% ====================================================================
\begin{agradecimentos}
Em primeiro lugar, agradeço a meus pais. Sem eles eu não estaria aqui.

Em segundo, agradeço a internet, cada site e cada pessoa que pôde me fazer
entender e conhecer algo que eu não sabia.
\end{agradecimentos}

\begin{epigrafe}
    \vspace*{\fill}
	\begin{flushright}
		\textit{``A única coisa que não muda é que tudo muda.''} \\
		(HERÁCLITO DE ÉFESO)
	\end{flushright}
\end{epigrafe}


\noindent STAVIS, Derek Willian. \textbf{\imprimirtitulo} Florianópolis, 2015.
\pageref{nropaginas}f. Trabalho de Conclusão de Curso Superior de Tecnologia em
Análise e Desenvolvimento de Sistemas. Faculdade de Tecnologia do SENAI,
Florianópolis, 2015.

\vspace{1cm}
\setlength{\absparsep}{18pt} % ajusta o espaçamento dos parágrafos do resumo
\begin{resumo}

  Os avanços trazidos nos últimos quatro anos no ambiente gráfico GNOME foram de
  suma importância para atingir uma experiência diferenciada e consistente de
  uso. Conceitos já difundidos no design de interfaces de usuário foram
  revisitados, paradigmas foram quebrados, mudando a forma com que os usuários
  vêem e interagem com o sistema operacional. Entretanto, nem todos os softwares
  disponíveis para a plataforma acompanharam a velocidade de desenvolvimento da
  plataforma GNOME, e muitos ainda necessitam de atenção. Com o lançamento do
  HIG (Guia de Interface com o Usuário) no GNOME versão 3.12 toda filosofia e
  padrões de design da plataforma foram sintetizados em linguagem simples e
  objetiva, tornando-se o material referência para o desenvolvimento e
  manutenção de seu ecosistema de softwares. Esta pesquisa aborda o processo de
  adaptação do cliente de torrent Transmission de acordo com o guia de
  interfaces do GNOME versão 3.14.

  \vspace{\onelineskip}

  \noindent
  \textbf{Palavras-chave}: Interfaces Gráficas. GTK+. GNOME. Design. Redesign.

\end{resumo}

% ====================================================================
% Abstract 
% ====================================================================
\noindent
STAVIS, Derek Willian. \textbf{Uma proposta de interface do cliente de torrent Transmission baseado no GTK HIG 3.14}
Florianópolis, 2013. 89f. Trabalho de Conclusão de Curso Superior de Tecnologia 
em Análise e Desenvolvimento de Sistemas. Faculdade de Tecnologia do SENAI,
Florianópolis, 2013.

\vspace{1cm}
\begin{resumo}[\textbf{ABSTRACT}]
 \begin{otherlanguage*}{english}
   This is the english abstract.

   \vspace{\onelineskip}
 
   \noindent 
   \textbf{Key-words}: Graphical User Interfaces. GTK+. GNOME. Design. Redesign.
 \end{otherlanguage*}
\end{resumo}



% lista de figuras
\pdfbookmark[0]{\listfigurename}{lof}
\listoffigures*
\clearpage

% inserir lista de tabelas
\pdfbookmark[0]{\listtablename}{lot}
\listoftables*
\clearpage

% Arquivos que devem ser alterados
% ====================================================================
% Siglas 
% ====================================================================

\begin{siglas}
  \item[GTK] GIMP Toolkit - Biblioteca de componentes para criação de interfaces gráficas.
  \item[GIMP] GNU Image Manipulation tool - Ferramenta de manipulações de imagem de código aberto, sob licensa GNU.
  \item[GNOME] Ambiente gráfico de estação de trabalho disponível para GNU/Linux.
  \item[GNU/Linux] Distribuição de software de código aberto que forma a base de um sistema operacional.
\end{siglas}


\include{content/simbolos}

% inserir o sumario
\pdfbookmark[0]{\contentsname}{toc}
\tableofcontents*
\clearpage

% ----------------------------------------------------------
% ELEMENTOS TEXTUAIS
% ----------------------------------------------------------
\textual

% PARTE - preparação da pesquisa
% ----------------------------------------------------------
%\part{Preparação da pesquisa}


% Informações que devem ser alteradas
% **************************************************
% ---
% Introdução
% ---
\chapter{INTRODUÇÃO}

O lançamento de um guia de padrões para o desenvolvimento de interfaces gráficas
orientadas ao ambiente gráfico GNOME foi de suma importância para alinhar
desenvolvedores ao intuito deste ambiente gráfico de código aberto.

Com esta adição, a plataforma GNOME passa a ter mais harmonia, e os aplicativos
padrão são os primeiros a ganharem este tratamento. Entretanto, aplicativos de
terceiros que já funcionavam na plataforma passam a apresentar-se defasados,
perdendo em usabilidade e em beleza.

Um aplicativo que sofre deste efeito é o Transmission, um cliente de torrent
muito popular, que foi desenvolvido na era GTK 2, onde os requisitos, recursos e
a visão da plataforma GNOME eram diferentes.

Esta pesquisa detalha o processo de pensamento por trás da adaptação deste
cliente de torrent, embasando as mudanças no guia de padrões e em aplicativos
já existentes na plataforma GNOME.

\section{OBJETIVOS}

\subsection{Objetivo geral}

Propor adaptações na interface gráfica GTK do cliente de torrent Transmission de
acordo com as recomendações do HIG versão 3.14, trazendo mais harmonia para seus
utilizadores na plataforma GNOME.

\subsection{Objetivos específicos}

\begin{enumerate}
  \item Identificar diferenças entre os padrões de design da versão 2.82 do
  Transmission e padrões propostos pelo HIG 3.14
  \item Propor adaptações na interface do Transmission beaseadas no HIG 3.14
  \item Enviar as adaptações propostas para o projeto de código aberto. 
\end{enumerate}

\section{METODOLOGIA}

Esta pesquisa possui duas frentes de trabalho:

Atraves de pesquisa exploratória bibliográfica serão levantados conhecimentos
sobre os principais componentes de interface do HIG 3.14 e sua utilização na
plataforma GNOME. Esta frente de trabalho deve compilar um conjunto de 
conhecimentos importantes para conhecer a abrangêcia do HIG e sua utilização. 

Um estudo de caso focado nos aplicativos padrão da plataforma GNOME formará uma
pequisa experimental, produzindo documentação visual sobre melhorias possíveis
na interface do Transmission.

\section{ESTRUTURA DO TRABALHO}

Explicar como o trabalho está estruturado.

% ---
% Capitulo de revisão de literatura
% ---
\chapter{REVISÃO DA LITERATURA}\label{cap-revisao}

O desenvolvimento de mecanismos que automatizam tarefas existe desde a idade das
pedras, onde hominídeos produziam ferramentas para auxílio próprio. A
transversalidade do conhecimento e da experimentação nos levou a descoberta de
novas metodologias e o aperfeiçoamento de técnicas, consequentemente modificando
a forma com que o ser humano interage com o mundo \cite[p.
1]{sartori2010neurociencia}.

\section{Interfaces gráficas}

O conceito de interface é extremamente amplo, e foi largamente difundido com o
início dos estudos de interação humano-máquina. O avanço da industrialização
permitiu com que máquinas extremamente complexas substituíssem seres humanos nas
tarefas mais difíceis,  deixando estes, seus operadores, apenas com a
responsabilidade de pilotá-las de forma simples e segura.

O advento de tecnologias multimídias interativas como computadores, celulares e
tablets elevou o patamar da criação de interfaces e gestão de tarefas ao estado
da arte, agregando conhecimentos transversais de artistas visuais a músicos.

\citeonline[p. 1]{ferreira2005semiotics} afirmam que a criação de “interfaces
com o usuário ainda está mais para arte do que para ciência.", e justificam,
explicando que "A maior parte do design ou redesign é baseado em estudos
empíricos ou protótipos, e ainda há muito pouca compreensão teórica ou de
engenharia de como conduzir o processo de design e produzir bons designs pela
primeira vez”.

O primeiro conceito científico de gerenciamento de tarefas através da
sobreposição de janelas data de 1969, na tese de Ph.D. de Alan Kay. A
implementação de seu conceito, vista pela primeira vez em funcionamento no
sistema do Xerox PARC, é largamente utilizada até hoje por grandes sistemas
tanto comerciais quanto de código aberto, como Windows, OS X, GNOME, KDE
\cite[p. 7]{myers2000past}.

Interfaces gráficas modernas foram alcançadas através da associação entre
hardware e software, utilizando de diversos dispositivos de entrada e saída,
além de  inúmeras camadas de software, com o intuito de promover usabilidade e
fácil adaptação.

Dispositivos de entrada transformam coordenadas do mundo real para o mundo
virtual, registrando condições externas que podem ser modificadas através da
interação com um ou mais atores. Dispositivos de entrada comuns são mouse, tela
de toque, teclado, etc.

Dispositivos de saída projetam informações geradas por um sistema computacional
e seu caráter é geralmente baseado nos sentidos: Visão, audição, tato. O
dispositivo de saída mais comum é o monitor, que tem por finalidade projetar
imagens compatíveis com a capacidade humana de visão.

\section{Gerenciadores de Janelas e Toolkits}

Gerenciadores de janelas suportam a separação da imagem de um dispositivo de
saída (geralmente monitores) em múltiplas regiões de desenho, comumente chamadas
de janelas  \cite[p. 5]{myers1996uimss}.

Um gerenciador de janelas moderno também tem por finalidade coordenar a exibição
de um conjunto de janelas em um conjunto de monitores, escutar por eventos de
entrada (mouse, teclado) e informar os responsáveis pelas janelas sobre
alterações no layout de tela (dimensões da tela, dimensões da janela, espaço de
cor).

Gerenciadores de janelas, porém, não tem por responsabilidade preencher as áreas
de desenho com gráficos, e como suas APIs operam geralmente a nível de pixel, a
tarefa de escrever um programa gráfico acaba sendo demorada e entediante.
\apudonline{rosenthal1988simple}{myers2000past}. Além disso, se cada
desenvolvedor criasse seus próprios componentes, seria praticamente impossível
disponibilizar uma experiência consistente ao usuário.

Para solucionar este problema ferramentas conhecidas como toolkits foram criadas
sobre as abstrações disponibilizadas pelos WMs. Sua finalidade é esconder
características de baixo nível, disponibilizando uma fachada homogênea e
portável para o desenvolvedor de software, além de comportamento e experiência
visual consistente para o usuário final \cite{myers2000past}.

As responsabilidades de um toolkit incluem desenhar elementos de interface
gráfica como texto, botões, imagens, barras de progresso, etc, de acordo com um
ou mais estilos visuais. Também é sua responsabilidade processar eventos de um
ou mais dispositivos de entrada (mouse, teclado, painel de toque) verificar a
colisão de um evento com elementos de interface gráfica (clique em um botão, por
exemplo), traduzir e informar os eventos para a aplicação proprietária da
janela.

Toolkits multiplataforma podem ser utilizados para escrever interfaces gráficas
portáveis, permitindo com que o mesmo código seja recompilado para um sistema
operacional diferente do em que foi escrito e funcione da mesma forma.

\section{A plataforma GNOME e o toolkit GTK}

Existem diversas opções de gerenciadores de janela de código aberto, comumente
incluídos em várias distribuições de Linux. O GNOME, um ambiente gráfico
bastante difundido pelos usuários de Linux, foi fundado e está em constante
desenvolvimento por uma comunidade de engenheiros de software ao redor do mundo.

Muito além de um mero gerenciador de janelas, a plataforma GNOME se desenvolveu
a ponto de ser constituída por uma série de aplicações base, incluindo um
gerenciador de janelas, um lançador de aplicações e diversos aplicativos
integrados, como calculadora, editor de texto, gerenciadores de arquivos, redes,
contatos, etc.

A plataforma GNOME e seus aplicativos integrados são altamente baseados no
toolkit GTK. De acordo com o site oficial, o ”GTK +, ou GIMP Toolkit, é um
conjunto de ferramentas multi-plataforma para criar interfaces gráficas de
usuário. Oferecendo um conjunto completo de widgets, o GTK + é adequado para
projetos desde pequenas ferramentas pontuais até suítes completas de
aplicativos.”  [1].

\section{O código aberto e desenvolvimento contínuo}

Uma das característica das plataformas de código aberto é a distribuição de
esforços em prol do constante desenvolvimento e melhoria. A pluralidade de
opiniões e idéias eleva o patamar das discussões e permite com que vários pontos
de vista sejam levados em consideração na evolução da plataforma.

Apesar dos prós existentes na distribuição de esforços também existem os contras
— Projetos que são desenvolvidos paralelamente nem sempre avançam na mesma
velocidade. E o contra fica mais sério quando um projeto depende do outro, como
é o caso de toolkits e programas que consomem suas APIs (Application Programming
Interfaces - Interfaces Programador-Programa).

Além dos aplicativos integrados, geralmente garantidos de acompanhar a evolução
da plataforma, sua ergonomia e visual, existe uma vasta gama de aplicações tanto
de código aberto quanto proprietárias disponíveis para atender as mais variadas
necessidades. Em sua grande maioria os aplicativos também são mantidos pela
comunidade, e se não atualizados podem ficar defasados em usabilidade, ergonomia
e consistência com a plataforma.

\section{Impactos na consistência e usabilidade}

Shneiderman, 1992 descreve a usabilidade como uma combinação das seguintes
características:

\begin{itemize}
    \item Facilidade de aprendizado
    \item Alta velocidade de operação
    \item Baixa taxa de erros
    \item Satisfação do usuário
    \itemRetenção de usuários pelo tempo
\end{itemize}

Em prol da usabilidade a maioria dos toolkits aplica um modelo próprio de
apresentação, organização, interação e estilização de widgets, que
costumeiramente varia de acordo com o tipo de ambiente onde é executado
(desktop, tablet e celular).

O conjunto de padrões de design recomendado ao utilizar o GTK para garantir a
máxima usabilidade da plataforma foi especificado oficialmente através de um
documento lançado no ano de 2014, acompanhando a versão 3.14 do toolkit, sob o
título de Human Interface Guidelines.

O HIG, como também é chamado, é uma literatura ilustrada de diretrizes
recomendadas no desenvolvimento de interfaces gráficas que utilizem o toolkit,
com o intuito de reforçar a consistência visual e integração com diferentes
gerenciadores de janela.

\section{Estudo de caso: Transmission}

Um aplicativo famoso disponível para a plataforma GNOME é o cliente de
BitTorrent Transmission. Aclamado pela sua simplicidade, o aplicativo é
utilizado para compartilhar arquivos através da internet.

A interface gráfica do Transmission foi desenvolvida na era GNOME 2, e por mais
que componentes básicos como botões e listas não tenham mudado agressivamente no
GNOME 3, muitos padrões de design foram criados ou mudaram, trazendo mais
conforto e consistência para os usuários e desenvolvedores do toolkit.

O objetivo principal desta pesquisa é explorar o HIG do GTK 3.14, além de
avançar o desenvolvimento de um software livre mantido pela comunidade de código
aberto, propondo e contribuindo melhorias visuais e de código-fonte na interface
gráfica do Transmission.

Esta pesquisa foi baseada na análise de diversos casos de uso de softwares
contidos na plataforma GNOME Desktop, versão 3.14, e tem a pretensão de
descrever o processo de pensamento e a motivação por trás das adaptações de
interface gráficas propostas.


% ---
% Procedimentos metodológicos
% ---
\chapter{PROCEDIMENTOS METODOLÓGICOS}

Descrevem-se, neste capítulo, os procedimentos metodológicos que nortearam a pesquisa.

% ---
% Resultados
% ---
\chapter{RESULTADOS E DISCUSSÕES}

Neste capítulo são apresentados os resultados da pesquisa descrita no capítulo \ref{cap-revisao}.

\begin{figure}[htb]
  \begin{center}
    \caption{\textbf{Figura de exemplo 2}}\label{fig-exemplo1}
    \includegraphics [scale=0.6]{logo-senai.jpg}
    \fonte{SENAI}\label{fig-exemplo1}
  \end{center}
\end{figure}
% ---
% Conclusão
% ---
\chapter{CONCLUSÃO}

As conclusão do trabalho são apresentadas aqui.

% **************************************************

% ----------------------------------------------------------
% ELEMENTOS PÓS-TEXTUAIS
% ----------------------------------------------------------
% \postextual

% ----------------------------------------------------------
% Referências bibliográficas
% Arquivos que devem ser alterados
\bibliography{content/referencias}

% ----------------------------------------------------------
% Glossário
% ----------------------------------------------------------
%
% Consulte o manual da classe abntex2 para orientações sobre o glossário.
%
%\glossary

% ----------------------------------------------------------
% Apêndices
% ----------------------------------------------------------

% Informações que devem ser alteradas
% **************************************************
% ---
% Inicia os apêndices
% ---
\begin{apendicesenv}

\chapter{Código fonte}
Código de minha autoria. O apêndice é opcional ao TCC e deve ser elaborado pelo próprio autor. Destina-se a complementar as ideias, sem prejuízo do tema do trabalho. Segue um exemplo:

\scriptsize
\begin{lstlisting}
#include <stdio.h>

int main() {
  printf("Ola mundo !\n");
  return 0;
}
\end{lstlisting}

\end{apendicesenv}

% ----------------------------------------------------------
% Anexos
% ----------------------------------------------------------
\begin{anexosenv}

\chapter{Pesquisa IBGE}
O anexo é opcional ao TCC e são informações não elaboradas pelo próprio autor, mas que tem como objetivo complementar as ideias, sem prejuízo do tema do relatório.

\end{anexosenv}

% Etiqueta para auxiliar contagem do numero de paginas do texto e dos elementos pos-textuais
\label{nropaginas}

% **************************************************

%---------------------------------------------------------------------
% INDICE REMISSIVO
%---------------------------------------------------------------------
\printindex

\end{document}
